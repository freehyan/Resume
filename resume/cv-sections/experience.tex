%----------------------------------------------------------------------------------------
%	SECTION TITLE
%----------------------------------------------------------------------------------------

\cvsection{项目经历}

%----------------------------------------------------------------------------------------
%	SECTION CONTENT
%----------------------------------------------------------------------------------------

\begin{hyancventries}

%------------------------------------------------

\hyanentrytitlestyle{基于极线采样(Epipolar Sampling)的户外光大气散射}\hspace{2cm}
\entrydatestyle{2015.09-2015.12}\hspace{3.25cm}
\entrylocationstyle{团队/组长}
\newline
\entrypositionstyle{{\color{red}项目描述}:大气散射是日常生活中基本现象,由于空气中微粒的影响,光线会产生不同的亮度和频谱上的变化,直观的效果是蔚蓝的天空和黄昏的日落。而在虚拟世界加入大气散射效果会使户外场景更加真实。此论文通过建立大气物理模型,预计算散射积分方程,使用极线采样和一维最小最大二叉树(1D Minimum/Maximum Binary Trees)方法加速,实时模拟出基于物理的高性能物理大气散射效果。}

\entrypositionstyle{{\color{red}项目职责}:主要负责框架的设计与实现,大气中参与介质的密度估计,散射积分离散化表达,以及极线采样模块。}
%------------------------------------------------

\hyanentrytitlestyle{HIVE游戏渲染引擎}\hspace{7.3cm}
\entrydatestyle{2016.01-2016.06}\hspace{3.25cm}
\entrylocationstyle{团队/组员}
\newline
\entrypositionstyle{{\color{red}项目描述}:游戏渲染引擎是指已经开发好并用于快速开发的游戏组件,目的是让图形编程人员更快速地实现虚拟游戏或仿真特效。HIVE针对OpenGL,DirectX,Vulkan不同的图形API渲染语言,设计了基于抽象层、实现层、应用层并具有良好扩展性的三层架构引擎。}

\entrypositionstyle{{\color{red}项目职责}:参与渲染引擎框架的设计与实现,负责文件系统模块(日志和配置文件),游戏场景图模块(模型变换、场景组织),绘制管线模块(Pass的执行、Shader的加载和验证、Uniform的更新),以及资源管理器模块。}
%------------------------------------------------

\hyanentrytitlestyle{基于光源链表(Light Linked List)的实时光照}\hspace{3.3cm}
\entrydatestyle{2016.07-2016.09}\hspace{3.25cm}
\entrylocationstyle{团队/组长}
\newline
\entrypositionstyle{{\color{red}项目描述}:延迟渲染是游戏中最受欢迎的实时渲染技术,它可以处理高几何复杂度场景和动态多光源问题,但是却不能处理透明物体和烟雾效果。LLL算法针对每个像素建立光源链表可以有效解决此问题,而且算法效率优于传统的延迟渲染技术。}

\entrypositionstyle{{\color{red}项目职责}:查阅相关论文和实现算法效果,加入OIT(Order Independent Transparency)算法优化最终效果,设计实验测试Deferred Shading和LLL在不同几何场景和光源数量的效率对比。}

%------------------------------------------------
\hyanentrytitlestyle{基于机器学习的并行绘制系统}\hspace{5.7cm}
\entrydatestyle{2016.10-2016.12}\hspace{3.25cm}
\entrylocationstyle{团队/组员}
\newline
\entrypositionstyle{{\color{red}项目描述}:针对基于sort-first结构的并行绘制系统中体系结构,任务划分,时序控制逻辑,和负载均衡等关键性问题提供了完整的解决方案。并行绘制系统通过机器学习负责负载均衡和InfiniBand负责网络传输模块,实现了复杂场景绘制的加速。}

\entrypositionstyle{{\color{red}项目职责}:参与系统的设计与实现,主要负责绘制模块,并协助其他小组完成功能模块,代码复查和性能优化。}

%------------------------------------------------


\end{hyancventries}