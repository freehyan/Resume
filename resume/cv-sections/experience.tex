%----------------------------------------------------------------------------------------
%	SECTION TITLE
%----------------------------------------------------------------------------------------

\cvsection{项目经历}

%----------------------------------------------------------------------------------------
%	SECTION CONTENT
%----------------------------------------------------------------------------------------

\begin{hyancventries}

%------------------------------------------------
\hyanentrytitlestyle{HIVE 渲染引擎}\hspace{5.9cm}
\entrydatestyle{2016.01-2016.06}\hspace{3.25cm}
\entrylocationstyle{团队/组员}
\newline
\entrypositionstyle{{\color{red}项目描述}:HIVE是针对图形程序员的开发框架, 设计了API无关的插件式结构。 延伸出抽象逻辑层,图形层,应用层三层引擎架构,支持多种不同的API(OpenGL, DirectX, Vulkan)绘制语言,具有良好的扩展性。}

\entrypositionstyle{{\color{red}项目职责}:参与渲染引擎框架的设计与实现,负责文件系统模块(日志和配置文件),游戏场景图模块(模型变换、场景组织),绘制管线
	模块(Pass的执行、 Shader的加载和验证、 Uniform的更新),以及资源管理器模块。}

%------------------------------------------------
\hyanentrytitlestyle{图形算法研究}\hspace{5.9cm}
\entrydatestyle{2016.07---*******}\hspace{3.25cm}
\entrylocationstyle{个人||团队}
\newline
\entrypositionstyle{{\color{red}项目描述}:研究和实现了图形学实时渲染相关领域的算法,其中包括延迟着色、 反走样、 顺序无关透明、 阴影、 流体渲染、 屏幕空间光泽反射,大气散射,体积雾等,期间也搭建了自己的渲染引擎Cooler。}

\entrypositionstyle{{\color{red}项目地址}:https://github.com/freehyan/Graphics 和 https://github.com/freehyan/Cooler}

%------------------------------------------------
\hyanentrytitlestyle{基于多种负载均衡方式的并行绘制系统}\hspace{2.1cm}
\entrydatestyle{2016.10-2016.12}\hspace{3.25cm}
\entrylocationstyle{团队/组员}
\newline
\entrypositionstyle{{\color{red}项目描述}: 对基于sort-first结构的并行绘制系统中体系结构,任务划分,时序控制逻辑,和负载均衡等关键性问题提供了完整的解决方案。 并行绘制系统使用时间帧相关性,机器学习等不同的负载均衡算法,Infliband的网络传输和渲染引擎的底层绘制,加速了复杂场景绘制.}

\entrypositionstyle{{\color{red}项目职责}:参与并行绘制框架的设计与实现,主要负责控制节点和绘制节点的流程,并协助其他小组完成相应的功能模块,以及代码把控和性能优化。}

%------------------------------------------------



\end{hyancventries}